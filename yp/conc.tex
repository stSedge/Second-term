Профессия разработчика баз данных является ключевой для эффективного управления информацией в современном мире. Спрос 
на нее будет продолжать расти в ближайшие годы, поскольку количество данных продолжает увеличиваться, а 
компании все больше заинтересованы в экспертах по их обработке и анализу. Ведь даже самоуправляемая база данных нуждается в администрировании, но уже более квалифицированными специалистами. 

Подводя итог, разработчик баз данных должен как минимум знать непосредственно теорию баз данных и свободно владеть языком запросов и управления SQL. Также на практике появится необходимость работать с большими данными, высоконагруженными системами, нереляционными базами данных, облачными технологиями и так далее.

По данным компании интернет"=рекрутмента HeadHunter\footnote{https://hh.ru} на июнь 2023 года в Саратове есть 9 вакансий на эту должность со средней заработной платой 90 тыс.р./месяц, а в Москве "--- 1003 вакансии, с заработной платой от 60 до 600 тыс.р./месяц. При этом, если переформулировать запрос как <<программист SQL>>, то количество вакансий вырастет в Саратове до 92, а в Москве "--- до 4959. Такой результат неудивителен, ведь умение работать с базами данных хотя бы на начальном уровне является базовым для IT"=специалиста. А по данным интернет"=портала Habr\footnote{https://habr.com} владение языком SQL входит в топ"=3 необходимых навыков для программиста.

Обобщая вышесказанное, данная специальность является одной из самых перспективных и высокооплачиваемых в IT"=индустрии.
