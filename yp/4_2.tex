Облачные вычисления являются наиболее популярной технологией в последнее время, поэтому неудивительно, что их стали
использовать и в проектировании баз данных. 

Облачная база данных (Cloud Database) "--- это служба базы данных, созданная и доступная через облачную платформу. Она выполняет те же функции, что и традиционная база данных, с дополнительной гибкостью облачных вычислений. Пользователи устанавливают программное обеспечение в облачной инфраструктуре для реализации базы данных.

Существует две основные модели развертывания облачной базы данных:
\begin{enumerate}
    \item Традиционная. Организация приобретает пространство виртуальной машины у поставщика облачных услуг, а база данных развертывается в облаке. 
    \item База данных как услуга (DBaaS). Организация заключает договор с поставщиком облачных услуг и база данных работает во внешней инфраструктуре. Модель DBaaS предоставляет организациям наибольшую ценность, позволяя им использовать внешнее управление базами данных, оптимизированное за счет автоматизации программного обеспечения, вместо того, чтобы нанимать собственных специалистов по базам данных.
\end{enumerate}

Новейший и наиболее инновационный тип облачной базы данных "--- это самоуправляемая (self"=driving) облачная база данных. В то время как локальным базам данных требуется специальный администратор баз данных для управления ими, такие же глубокие знания не требуются для управления автономной облачной базой данных. Этот тип облачной базы данных использует машинное обучение для автоматизации настройки, обеспечения безопасности, резервного копирования, обновления и других рутинных задач управления, которые традиционно выполнялись администраторами баз данных.

Исследование IDC показывает, что до 75\% общих затрат предприятия на управление данными могут быть связаны только с оплатой труда. Самостоятельная база данных потенциально может сэкономить среднему предприятию сотни или, возможно, тысячи рабочих часов сотрудников, занятых полный рабочий день, ежегодно для каждой из основных корпоративных баз данных. Кроме того, было подсчитано, что 72\% корпоративных ИТ-бюджетов уходит на поддержку существующих систем, а на инновации остается лишь 25\%.

Самостоятельные базы данных могут иметь большое значение для устранения этих высоких затрат и предоставления предприятиям возможности использовать своих администраторов баз данных для более важных задач, таких как моделирование данных, помощь программистам с архитектурой данных и планирование будущих мощностей \cite{8}.