XXI век по праву считается веком информации и информационных технологий. Трудно
представить сферу деятельности, в которой ни коем образом не приходится работать с
большими объемами данных. Исходя из этого, мы закономерно сталкиваемся со
следующими задачами: в каком виде хранить, как быстро искать и обрабатывать, как
распространять между различными пользователями информацию, обеспечивая при этом
её безопасность и целостность.

Наиболее эффективным ответом на перечисленные вопросы является создание базы
данных. База данных "--- структурированное поименованное хранилище
информации. Они используются в науке, здравоохранении, образовании,
бизнесе и не только.

Базы данных могут быть разработаны на различных языках программирования,
включая Java, Python, C++, Ruby, PHP, JavaScript и другие. Кроме того, 
существует множество специализированных языков запроса и управления данными,
таких как: SQL, NoSQL или MongoDB. Но, очевидно, недостаточно
спроектировать базу данных. Её необходимо поддерживать, обновлять, чистить, каким"=то
образом к ней обращаться.

Неудивительно, что разработчик баз данных "--- это довольно востребованная
специальность на рынке IT"=специалистов. Давайте подробнее рассмотрим, что
входит в обязанности такого сотрудника и какими знаниями и навыками он должен
обладать.