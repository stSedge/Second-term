Cистема баз данных "--- это компьютеризированная система
хранения записей, т.е. компьютеризированная система, основное назначение
которой хранить информацию, предоставляя пользователям средства её извлечения
и модификации.

Любая система баз данных состоит из 4 основных частей: данные, аппаратное обеспечение,
пользователи и программное обеспечение. Наиболее интересным является последний из перечисленных компонент. 

Между физической базой данных (т.е. данными, которые хранятся на компьютере) 
и пользователями системы располагается уровень программного
обеспечения, который можно называть по"=разному: диспетчер базы данных (database
manager), сервер базы данных (database server) или, что более привычно, система
управления базами данных, СУБД (DataBase Management System, DBMS). 

Все запросы пользователей на получение доступа к базе данных обрабатываются СУБД. 
Все имеющиеся средства добавления файлов (или таблиц), выборки и обновления данных в
этих файлах или таблицах также предоставляет СУБД. Основная задача СУБД "--- дать
пользователю базы данных возможность работать с ней, не вникая во все
подробности работы на уровне аппаратного обеспечения.

СУБД "--- это наиболее важный, но не единственный программный компонент системы. В
числе других можно назвать утилиты, средства разработки приложений,
средства проектирования, генераторы отчетов и диспетчер транзакций. 

Наиболее распространенными являются две СУБД "--- MySQL и Oracle. Зачастую, саму СУБД также называют
базой данных, из"=за чего возникает путаница в терминах. Это, безусловно,
неправильно, но ошибка широко распростронена \cite{date}.