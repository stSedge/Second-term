Большинство систем баз данных описывается архитектурой, предложенной исследовательской группой
группой ANSI/SPARC (Study Group on Data Management Systems), так называемой
архитектурой ANSI/SPARC.

Архитектура ANSI SPARC включает три уровня:
\begin{enumerate} 
\item Внутренний уровень (физический) наиболее близок к физическому хранилищу информации, 
т.е. связан со способами сохранения информации на физических устройствах.

\item Внешний уровень (пользовательский логический) наиболее близок к пользователям, 
т.е. связан со способами представления данных для отдельных пользователей.

\item Концептуальный уровень (общий логический или просто логический) является <<промежуточным>>
уровнем между двумя первыми.
\end{enumerate}

Если внешний уровень связан с индивидуальными представлениями пользователей, то
концептуальный уровень связан с обобщенным представлением пользователей. Большинству
пользователей нужна не вся база данных, а
только ее небольшая часть, поэтому может существовать несколько внешних
представлений, каждое из которых состоит из более или менее абстрактного
представления определенной части базы данных, и только одно концептуальное
представление, состоящее из абетрактного представления базы данных в целом \cite{3}.