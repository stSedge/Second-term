Одно из требований к организации базы данных "--- обеспечение возможности отыскания одних сущностей по значениям других.

Связь (relationship) "--- это ассоциирование сущностей. Сущности, участвующие в
связи, называются участниками (participants). Связи не могут существовать без связываемых сущностей. Они могут быть бинарными, тернарными, N-арными и рекурсивными.

В качестве названия связи
используется глагол в активной или пассивной форме. Например, <<студент (STUDENT)
занимается в группе (CLASS)>> или <<преподаватель (PROFESSOR) ведет занятия в
группе (CLASS)>>.

Связи между сущностями всегда действуют в обоих направлениях, поэтому связь трудно классифицировать, если известна только одна ее сторона \cite{5}.