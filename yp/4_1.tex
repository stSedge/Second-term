Стремительно растущий объем информации ставит перед нами новые сложные задачи по организации ее хранения и обработки.
По по оценкам IDC размер <<цифровой вселенной>> в 2006 году составлял 0.18 зеттабайт, а к 2011 году должен был достигнуть 1.8 зеттабайт, продемонстрировав десятикратный рост за 5 лет. 

Закономерно появление термина <<Big Data>> или <<большие данные>> — это структурированные или неструктурированные массивы данных большого объема. Большие данные имеют четыре основные характеристики: объем, разнообразие, скорость и ценность. 

Классическая реляционная архитектура не подходит для работы с большими данными. Попытки приспособить реляционную СУБД к работе с большими данными приводят к следующему:

\begin{enumerate}
    \item Отказ от строгой согласованности.
    \item Уход от нормализации и внедрение избыточности.
    \item Потеря выразительности языка SQL и необходимость моделировать часть его функций программно.
    \item Существенное усложнение клиентского программного обеспечению.
    \item Сложность поддержания работоспособности и отказоустойчивости получившегося решения.
\end{enumerate}

Производители реляционных СУБД осознают все эти проблемы и уже начали предлагать масштабируемые
кластерные решения. Однако стоимость внедрения и сопровождения подобных решений зачастую не окупается.

Тогда почему бы не спроектировать архитектуру,
способную адаптироваться к возрастающим объемам данных и эффективно их
обрабатывать? Подобные мысли привели к появлению движения NoSQL.

NoSQL "--- это семейство баз данных, которые отличаются от традиционных реляционных баз данных (SQL) тем, что они не используют таблицы, схемы и SQL"=язык запросов. Вместо этого они используют другие модели данных, такие как документы, графы или ключ"=значение.  

Среди самых популярных NoSQL баз данных можно выделить MongoDB, Cassandra, Redis и Couchbase. Они используются в различных областях, от веб"=разработки и аналитики больших данных до хранения данных в облаке.

NoSQL не подразумевает бездумного отказа от всех принципов реляционной модели. Более того, термин <<NoSQL>>
впервые был использован в 1998 году для описания реляционной базы данных, не использовавшей SQL. 

Популярность NoSQL стал набирать в 2009 году, в связи с появлением большого количества веб"=стартапов, для которых важнейшей задачей является поддержание постоянной высокой пропускной способности хранилища при неограниченном увеличении объема
данных. 

Рассмотрим основные особенности NoSQL подхода:

\begin{enumerate}
    \item Исключение излишнего усложнения. 
    \item Высокая пропускная способность. 
    \item Неограниченное горизонтальное масштабирование. 
    \item Рост роизводительности засчет пренебрежения согласованностью данных.
\end{enumerate}

Но все же, при выборе инструментария необходимо отталкиваться от задачи, поэтому нельзя сказать, что NoSQL может полностью заменить реляционную модель \cite{9}.