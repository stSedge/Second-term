Поговорим о том, как проектируются базы данных на самом верхнем
уровне "--- концептуальном, когда рассматривается только суть хранимых данных, их
свойства и связи между различными элементами без привязки к особенностям
физической реализации и конкретной СУБД.

Словарь Вебстера определяет модель так: <<модель "--- это
описание или аналогия, используемая для визуализации чего"=либо, что не может
наблюдаться непосредственно. Основное предназначение модели облегчить понимание
структур данных и их свойств, связей и ограничений.

Проектировщик БД использует модели данных как средство коммуникации, обеспечивающее
взаимодействие проектировщиков, прикладных программистов и конечных
пользователей. Если модель данных тщательно проработана, то она поможет лучше
понять положение дел в организациию.

Рассмотрим наиболее популярную модель <<Сущность"=связь>> и подробно разберем, как описываются в терминах этой
модели объекты (сущности), их атрибуты и связи\cite{4}.