Модель <<сущность"=связь>> используется для проектирования баз данных, которые затем могут быть реализованы в виде реляционной, иерархической, сетевой или иной модели данных. Реляционная модель является одной из самых популярных среди современных информационных систем. 

Существует строгое определение реляционной базы данных и теоретический фундамент для операций, которые
могут быть выполнены над ней. Однако в данном случае более полезно неформальное определение.

Реляционной называется база данных, в которой все данные, доступные пользователю, организованы в виде таблиц, а все операции базы данных выполняются над этими таблицами. 

Связи между данными устанавливаются с помощью ключей. Первичный ключ (Primary key) "--- это уникальный идентификатор каждой записи в таблице реляционной базы данных. Внешний ключ (Foreign key) "--- это поле или группа полей в таблице, которые связывают ее с другой таблицей. Он ссылается на первичный ключ другой таблицы и используется для создания связей между таблицами. Существуют и другие типы ключей \cite{10}.