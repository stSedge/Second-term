\documentclass[titlepage, 12pt]{article}

\usepackage[T2A]{fontenc}
\usepackage[english, russian]{babel}
\usepackage[utf8]{inputenc}
\usepackage{graphicx} % Required for inserting images

\begin{document}

\begin{titlepage}
    \newpage
    \begin{center}
    {\bfseries Саратовский национальный исследовательский государственный университет имени Н. Г. Чернышевского}
    \vspace{1cm}
    
    Кафедра математической кибернетики и компьютерных наук
    \vspace{6em}
    
    Стюхина Ангелина Сергеевна
     
     151 группа
    \end{center}

    \vspace{1.2em}

    \begin{center}
    \Large Молекулярный компьютер
    \end{center}

    \vspace{5em}

    \begin{center}
    %\Large
     реферат
     \end{center}
    \vspace{6em}

    \begin{center}
    Научный руководитель \\
    \>к.ф.-м.н., Машников В.В.\\
    \end{center}


    \vspace{\fill}

    \begin{center}
   Саратов 2023
    \end{center}

    \end{titlepage}

\setcounter{page}{2}
\tableofcontents
\newpage

\section*{Вступление}
\addcontentsline{toc}{section}{Вступление}
Среднестатистический пользователь не задумывается о том, как работает, а уж тем более, из чего сделан его кремниевый компьютер. И о том, что он кремниевый, тоже. Выбирая технику, он обращает внимание на технические характеристики, но не задаётся вопросом, за счёт чего процессор нового поколения стал мощнее старого и есть ли предел. А предел, очевидно, есть. Уже более 30 лет перед человечеством стоит вопрос о совпадении или несовпадении классов P и NP (одна из семи задач тысячелетия) и о невозможности решать NP-полные задачи за полимиальное время на современных компьютерах. Забегая вперед, скажем, что это всё-таки возможно, но не путём поиска наиболее эффективного алгоритма.

Бурное развитие техники, то есть современной микро- и наноэлектроники, происходит, в основном, за счёт постоянного уменьшения размеров элементов микросхем и увеличения плотности их размещения на кристалле. Это позволяет повышать скорость переработки информации. Однако неуклонное возрастание сложности и быстродействия интегральных схем на основе кремния не может продолжаться до бесконечности. На этом пути встаёт барьер, обусловленный фундаментальными законами физики.

В 1965 г., на заре компьютерной эры, директор отдела исследовательской компании Fairchild Semiconductors Гордон Мур предсказал, что количество транзисторов на одной микросхеме будет ежегодно удваиваться. Прошло почти полвека, но закон Мура по-прежнему актуален и работает.

Со временем практика микроэлектронного производства внесла в него поправку: сегодня считается, что удвоение числа транзисторов происходит каждые 18 месяцев. Такое замедление роста вызвано усложнением архитектуры компьютерной аппаратуры. Однако для кремниевой технологии производства закон Мура не может выполняться вечно.

Первое, наиболее очевидное ограничение на закон Мура связано с неконтролируемым ростом стоимости производства все более сложных устройств. За последние 30 лет оборудование для выпуска микросхем подорожало примерно в 1000 раз.

Но есть и другое, принципиальное ограничение на закон Мура. Возрастание плотности размещения элементов па микросхеме достигается за счет уменьшения их размеров. Уже сегодня расстояние между элементами процессора может составлять 0,13*10**(–6) м. Когда размеры транзисторов и расстояния между ними достигнут нескольких десятков нанометров, вступят в силу так называемые размерные эффекты – физические явления, полностью нарушающие работу традиционных кремниевых устройств. С уменьшением толщины диэлектрика в полевых транзисторах растет вероятность прохождения электронов через него, что также препятствует нормальной работе приборов.

Многие специалисты связывают будущее кремниевой электроники с переходом к трёхмерной архитектуре микросхем, позволяющей при тех же размерах элементов разместить гораздо большее их число в одном кристалле кремния (в применяемой сейчас планарной технологии все элементы микросхемы располагаются в одной плоскости на поверхности кристалла). Однако переход на трехмерную технологию не может принципиально решить упомянутую выше проблему размерных эффектов.

Еще один путь повышения производительности – применение вместо кремния и германия других полупроводников, например, арсенида галлия (GaAs) и других. За счет более высокой подвижности электронов в этом материале можно увеличить быстродействие устройств еще на порядок. Однако технологии на основе арсенида галлия намного сложнее кремниевых технологий, а, следовательно, их стоимость на порядки выше, что неприемлемо как массового производства, так и для пользователя.

Можно с уверенностью сказать, что современная технология создания вычислительных систем изживает себя. Микропроцессоры последних поколений содержат огромное число транзисторов (10 млн. и более). Можно уменьшать физические размеры транзисторов и интегральных схем, применяя нанотехнологии, но всему есть предел. Это лишь малая часть огромной проблемы, уже вставшей перед специалистами в сфере компьютерных технологий, проблемы приближения к пределу быстродействия. За последнее время вся числовая, текстовая, графическая, звуковая, видео и другая информация была перенесена на компьютерные носители. Теперь, когда в базах данных находятся сотни миллионов записей, поиск в них стал требовать гораздо большего времени, и новые алгоритмы поиска ненамного его уменьшают.

Стало очевидно, что необходимы компьютеры новых поколений с более высокими скоростными характеристиками. Поэтому специалисты всего мира взялись за решение этой проблемы путем создания вычислительной системы будущего.


\section{Молекулярный компьютер}

\subsection{Основные принципы работы}
Один из путей решения проблемы предлагает молекулярная электроника, или молетроника. Достижения в этой области позволяют говорить о молекулярном компьютере.

Что такое молекулярный компьютер? Это устройство, в котором вместо кремниевых чипов, применяемых в современных компьютерах, работают молекулы и молекулярные ансамбли. В основе новой технологической эры лежат так называемые «интеллектуальные молекулы». Такие молекулы могут существовать в двух термодинамически устойчивых состояниях, каждое из которых имеет свои физические и химические свойства. Переводить молекулу из одного состояния в другое (переключать) можно с помощью света, тепла, химических агентов, электрического и магнитного поля и т.д. Фактически такие переключаемые бистабильные молекулы — это наноразмерная двухбитовая система, воспроизводящая на молекулярном уровне функцию классического транзистора.

Особенно интересны такие превращения бистабильных молекул, после которых сильно меняется электронная конфигурация. Например, после изомеризации в молекуле образуется единая сопряжённая электронная система, следовательно, появляется способность проводить электрический ток.

Ещё в 1959 году Ричард Фейнман указал на то, что молекулы, обладающие определёнными свойствами, смогут работать как переключатели и заменить собой транзисторы („Химия и жизнь“, 2002, № 12). Это предсказание начинает сбываться. Размеры будущего молекулярного транзистора будут на два порядка меньше самых миниатюрных кремниевых. Поскольку производительность компьютера пропорциональна количеству транзисторов, размещаемых на единице площади, то выигрыш в производительности будет огромным. Так, если уменьшить размер транзистора до молекулярных размеров (примерно до одного нанометра), то на единице площади интегральной схемы поместится в миллион раз больше транзисторов. Если ещё вдобавок к этому время отклика уменьшится до фемтосекунд (на шесть порядков) — а именно таково характеристическое время протекания элементарной стадии химической реакции, — то эффективность молекулярного компьютера может оказаться в 100 миллиардов раз выше, чем современного кремниевого.


\subsection{Архитектура молекулярного компьютера}
Архитектура каждого компьютера включает три основных элемента: переключатели, память, соединяющие провода. Все элементы в молекулярных компьютерах будут отличаться от их же аналогов в нынешних вычислительных устройствах. Бистабильные молекулы — переключатели будут управляться световыми и электрическими импульсами или электрохимическими реакциями. Память может работать на принципе «запоминания» оптических или магнитных эффектов, а проводниками могут стать нанотрубки или сопряжённые полимеры. Сейчас уже созданы многочисленные варианты всех основных составляющих компьютера будущего. Рассмотрим их по отдельности.

\subsubsection{Переключатели}
Наиболее эффективные молекулярные переключатели основаны на фотохромных соединениях, которые изомеризуются при переходе в высшие возбуждённые электронные состояния. Это может быть процесс цис-транс-изомеризации, перициклических превращений, фотопереноса протона. После переключения кардинально перестраивается электронная конфигурация системы, а её геометрия остаётся практически прежней. 

Перспективны также топологические изомеры супрамолекул — например, переключатель, описанный Д.Ф. Стоддардом и Д. Хисом, которые сотрудничают с фирмой <<Хьюлетт Паккард>>. Монослой молекул катенана помещают между металлическим и кремниевым электродами. После электрохимического окисления супрамолекулы на одной из её частей появляется дополнительный положительный заряд. Поскольку в исходной форме эта часть соседствует с одноимённым зарядом, то после окисления плюсы отталкиваются и молекула перегруппировывается. Образуется вторая стабильная форма, и меняется электрическое сопротивление. Главное достоинство такого переключателя — его исключительно высокая устойчивость. Цикл окисления-восстановления катенана можно совершать 10-20 тысяч раз без заметного разрушения супрамолекулярной системы.

\subsubsection{Память}
Молекулярная память "--- разновидность памяти, запись информации и считывание которой производится на молекулярном уровне.
В основе ее существования лежат исследования сканирующего туннельного микроскопа. Данные в такую память можно внести, ориентировав молекулы в заданном направлении с помощью электромагнитного поля.

В настоящее время применяют магнитные и оптические носители памяти, которые основаны на принципе двумерной записи, и это ограничивает объёмы записываемой информации. Стандартный диск CD-ROM диаметром 12 см может содержать примерно 0,5 гигабайт данных. Теоретическая плотность оптической записи информации обратно пропорциональна квадрату длины волны используемого для записи света, поэтому предел возможностей однослойного компакт-диска равен 3,5·10**8 бит/см2 (для света с длиной волны 532 нм).

Чтобы записать информацию в объёме образца или, по крайней мере, на нескольких его слоях, нужна новая система записи. Для этого используют метод двухфотонного поглощения. Необходимая для записи энергия ($hv$) доставляется двумя фокусируемыми в нужной точке лазерными пучками с частотами $v1$ и $v2$, подобранными так, чтобы $hv = hv1 + hv2$. Впервые принципиальную возможность такой схемы показал П. Рентцепис (Калифорнийский университет) в конце 80-х годов XX века. Он использовал для этого, в частности, фотохромную спиропирановую систему. Поглотив два фотона, молекула А перегруппируется в окрашенную мероцианиновую форму В. Считывание записанной таким образом информации происходит при регистрации флуоресценции молекулы В, также возбуждаемой двухквантовым переходом. Флуоресценция — не единственный, но в силу особенно высокой чувствительности наиболее привлекательный метод считывания записанной информации.

Другой перспективный подход к созданию молекулярной памяти продемонстрировали М. Рид (Йельский университет) и Д. Тур (компания <<Хьюлетт Паккард>>). Они сделали сэндвич примерно из 1000 молекул ароматического дитиола и поместили его между золотыми электродами. При определённом напряжении, поданном на электроды, этот сэндвич удерживает электроны (то есть хранит данное состояние в памяти) в течение 10 минут (стандартная кремниевая динамическая память DRAM удерживает всего на миллисекунды). При напряжении 5В ученым удалось поддерживать ток в 0,2 микроампера, что соответствует потоку 1012 электронов в секунду. Это намного больше того, что они ожидали после теоретических расчётов. Интересно, что электроны проходят через молекулу без рассеяния тепла. Авторы исследования думают, что их <<электронная присоска>>, как они её назвали, может служить прототипом нового поколения динамической памяти.


\subsubsection{Соединительные провода}
Проводники обеспечивают сообщение между молекулярными транзисторами и молекулярными устройствами памяти. Дизайн проводников, также имеющих наноскопические размеры, учёные ведут по трём основным направлениям. Первое — это проводящие полимеры: допированный полиацетилен (Нобелевская премия 2000 года), политиофен, полианилин и др. Второе — различные органические проводники, которые обладают достаточно высокой проводимостью, до 102-103 с/м. Все они представляют собой длинные сопряжённые молекулы, в которых электрон переносится по цепи $\pi$-связей. 

Если к концам такой сопряжённой цепи присоединить металлсодержащие группы, то окисление или восстановление одной из них обеспечит достаточную проводимость по всей цепи. Комбинируя допированные (проводящие) и недопированные (со свойствами изоляторов или полупроводников) участки полимеров, можно получать электрические контуры с нужными свойствами.

Особые надежды возлагаются на третий тип проводников — нанотрубки. Это великолепный материал для молекулярной электроники. Нанотрубки с однослойными или многослойными стенками получаются при прохождении электрического разряда между двумя графитовыми электродами. Длина одностенных нанотрубок может достигать микрометров (диаметр около 1 нм), причём на отрезках по 150 нм сохраняются металлические свойства. Углеродные или боразотные нанотрубки можно заполнять металлами и получать таким образом одномерные проводники, состоящие из цепочек атомов металлов. 

С одностенными нанотрубками удается сделать еще более интересные вещи. При помощи атомно-силового микроскопа, скручивая однослойную нанотрубку, удалось получить участки, на которых сопротивление достигает 50 килоОм, в результате чего образуется барьер для движения электрона. При определённом напряжении можно переключать состояния одностенной нанотрубки: <<проводимое>>—<<непроводимое>>, перемещая один-единственный электрон. Фактически это прототип транзистора на одном электроне. Существует также прототип транзистора на одной молекуле, который изучают в Корнельском и Гарвардском университетах.



\subsection{Перспективы развития}
Молекулярные транзисторы, память и проводники — три составные части будущего молекулярного компьютера, и в их создании по отдельности есть значительные успехи. Но самая сложная задача — собрать все компоненты в работающее устройство. До её решения ещё далеко. Однако путь, по которому надо идти, вполне ясен: это принцип молекулярного распознавания, ответственный за самосборку и самоорганизацию сложных ансамблей и агрегатов молекул. Пока эта задача не решена, учёные предполагают делать гибридные устройства, сочетающие достоинства молекулярного подхода с наиболее успешными технологическими вариантами, найденными для кремниевых технологий. Гибридные устройства можно сделать, например, используя повышенное сродство атомов серы в органических молекулах к тяжёлым металлам, особенно золоту. Так создаются контакты между металлическими электродами и молекулярными проводниками.

Мысль учёных идет дальше. До сих пор мы рассматривали примеры, когда все функции компонентов компьютера обеспечиваются передвижением электронов в сложных молекулярных ансамблях. Между тем эти функции могут взять на себя и фотоны. Уже предложены различные варианты фотонных устройств, например молекулярный фотонный транзистор. В фотонном транзисторе фрагмент молекулы, поглощающий квант света (дипиррилбородифторид), играет роль стокового электрода, следующая молекула (цинковый порфирин) — проводника, а последний излучающий порфириновый фрагмент молекулы соответствует электроду истока. Магниевый порфирин работает как управляющий электрод — затвор. Если окислить этот затвор, то после поглощения света перенос энергии происходит не на цинковый порфирин, а на неизлучающий магниевый. В компьютерах на подобных транзисторах, регулирование всей его работы будет происходить с помощью света.

Вот в общих чертах то, что ждёт нас в ближайшем будущем. Это не значит, что после создания молекулярного компьютера существующее поколение кремниевых компьютеров полностью и сразу отомрёт, просто рядом с ним появится более мощная генерация. А что потом? Спинтроника и компьютеры на квантовых точках, ДНК-компьютеры.

\section{Практическое применение}
На данный момент активно ведутся исследования и работы в области молетроники. Рассмотрим некоторые интересные, на мой взгляд, примеры.

Так, французские ученые из Института Садрона в 2017 году успешно закодировали и затем прочитали слово Sequence (с англ. последовательность) (оно было представлено в ASCII-коде) с помощью последовательности синтетических полимеров. Таким образом, они доказали, что в молекулах полимеров можно хранить информацию, и занимать она будет в 100 раз меньше места (физического), чем на обычных жестких дисках. 


Чтобы закодировать информацию в полимеры, используются два разных типа мономеров («битов»), содержащих фосфатные группы. Первый тип обозначает единицу, а второй — ноль. Через каждые восемь мономеров идет молекулярный разделитель NO-C (сепаратор), отмечающий байт.

Чтобы расшифровать информацию, каждый байт сперва «отделяется» в месте расположения сепаратора. После этого фосфатные связи между мономерами уничтожаются, а сами мономеры идентифицируются с помощью масс-спектрометра.

Сейчас на то, чтобы закодировать и прочитать информацию, уходит несколько часов. Но по мнению ученых, проблема решаема — для этого нужно автоматизировать синтез полимеров и анализ последовательностей.

Следующей целью ученых является создание первой «молекулярной дискеты» — молекулы большего размера. Она сможет хранить несколько килобайт информации, например, страницу текста.

Есть успехи и в области молекулярных вычислений. Для доказательства возможности вычислений на базе ДНК Адельман выбрал проблему отыскания Гамильтонова пути графа (Leonard M. Adleman, Molecular Computation of Solutions to Combinatorial Problems, Science, № 266/94, с. 1021). Суть ее в следующем. Имеется граф с n вершинами, соединенными однонаправленными ребрами. Нужно найти путь из одной заданной вершины в другую, проходящий через все остальные вершины только один раз, или доказать отсутствие такого пути. Задача дискретная, решать ее можно лишь перебором, она имеет прикладное значение, и все существующие для нее детерминированные алгоритмы имеют экспоненциальное время исполнения.

В основе рассуждений Адельмана лежала возможность закодировать каждую вершину графа уникальной последовательностью олигонуклеотидов. (Сам алгоритм достаточно сложен и в его описании используются нетривиальные биохимические понятия, поэтому в данной работе мы не будем на нём останавливаться, но вы можете ознакомиться с ним в трудах Адельмана.) На выполнение всех операций у Адельмана ушло в 1994 г. семь рабочих дней. Самым трудоемким оказалось разделение молекул с помощью магнита. Вскоре другие авторы показали, как, используя аналогичный алгоритм, решать задачу Гамильтона для ребер, обладающих весом, "--- для этого вес нужно задавать целым числом и кодировать его в ДНК, повторяя нужное число раз последовательность, отображающую исходящую вершину ребра.

Подводя итог, инструменты молекулярной биологии были использованы для решения примера задачи о направленном гамильтоновом пути. Небольшой граф был закодирован в молекулах ДНК, а «операции» вычисления были выполнены со стандартными протоколами и ферментами. Этот эксперимент демонстрирует возможность проведения расчетов на молекулярном уровне.


\newpage
\section*{Заключение}
\addcontentsline{toc}{section}{Заключение}
Если компьютерные технологии продолжат развиваться с той же скоростью, с какой они делают это в наши дни, буквально через десять лет можно ожидать, что компьютеры станут в 1000 раз более мощными. Жесткие диски смогут в 10000 раз хранить больше информации. И вполне вероятно, что этот прорыв будет связан именно с молекулярными технологиями, а не с кремниевыми микросхемами, которые уже достигают предела своих возможностей.

Но, несмотря на то, что в области молекулярной электроники совершен ряд прорывов, фото молекулярного компьютера в сети Интернет найти не удастся. Это потому, что пока еще не существует самого компьютера на такой технологии.

Но уже в ближайшем будущем можно ожидать изобретения молекулярных компьютеров. Они принадлежат архитектуре фон Неймана, в этом уже можно быть уверенными сейчас. Это объясняется тем, что молекулы должны заменить электронные компоненты, а структура компьютера пока останется неизменной.

Только представьте, какой резкий скачок в развитии всех сфер нашей жизни ждёт человечество после создания и распространения молекулярных компьютеров, дающих возможность, как минимум, решать NP-полные задачи. Будем надеяться, что увидим всё это на нашем веку и, может, даже примем участие в разработке.





\newpage
\section*{Список использованных источников}
\addcontentsline{toc}{section}{Список использованных источников}


\begin{enumerate}
\item Минкин, В.И. Молекулярные компьютеры // <<Химия и жизнь>>. "--- №2. "--- 2004.
\item Черкесова, Л. В. Проблемы современной фундаментальной науки: учеб. пособ. "--- 2016.
\item Черногорова, Ю. В. Перспективы развития молекулярных вычислений // Молодой ученый. "--- 2016. "--- № 1 (105). "--- С. 47-49.
\item Борисов В. ДНК "--- основа вычислительных машин // PCWeek/RE. "--- 1999.
\item Abdelaziz Al Ouahabi, Jean-Arthur Amalian, Laurence Charles, Jean-François Lutz «Mass spectrometry sequencing of long digital polymers facilitated by programmed inter-byte fragmentation» "--- 2017.
\end{enumerate}

\end{document}
