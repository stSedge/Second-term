\documentclass[bachelor,och,coursework,times]{SCWorks}
% Тип обучения (одно из значений):
%    bachelor   - бакалавриат (по умолчанию)
%    spec       - специальность
%    master     - магистратура
% Форма обучения (одно из значений):
%    och        - очное (по умолчанию)
%    zaoch      - заочное
% Тип работы (одно из значений):
%    coursework - курсовая работа (по умолчанию)
%    referat    - реферат
%  * otchet     - универсальный отчет
%  * nirjournal - журнал НИР
%  * digital    - итоговая работа для цифровой кафдры
%    diploma    - дипломная работа
%    pract      - отчет о научно-исследовательской работе
%    autoref    - автореферат выпускной работы
%    assignment - задание на выпускную квалификационную работу
%    review     - отзыв руководителя
%    critique   - рецензия на выпускную работу
% Включение шрифта
%    times      - включение шрифта Times New Roman (если установлен)
%                 по умолчанию выключен
\usepackage{preamble}

\begin{document}

% Кафедра (в родительном падеже)
\chair{математической кибернетики и компьютерных наук}

% Тема работы
\title{ЛЕКСИЧЕСКИЙ И СИНТАКСИЧЕСКИЙ АНАЛИЗ ВЫРАЖЕНИЙ}

% Курс
\course{2}

% Группа
\group{251}

% Факультет (в родительном падеже) (по умолчанию "факультета КНиИТ")
%\department{факультета компьютерных наук и информационных технологий}

% Специальность/направление код - наименование
% \napravlenie{02.03.02 "--- Фундаментальная информатика и информационные технологии}
% \napravlenie{02.03.01 "--- Математическое обеспечение и администрирование информационных систем}
% \napravlenie{09.03.01 "--- Информатика и вычислительная техника}
\napravlenie{09.03.04 Программная инженерия}
% \napravlenie{10.05.01 "--- Компьютерная безопасность}

% Для студентки. Для работы студента следующая команда не нужна.
%\studenttitle{Студентки}

% Фамилия, имя, отчество в родительном падеже
\author{Рыданова Никиты Сергеевича}

% Заведующий кафедрой 
\chtitle{доцент, к.\,ф.-м.\,н.}
\chname{С.\,В.\,Миронов}

% Руководитель ДПП ПП для цифровой кафедры (перекрывает заведующего кафедры)
% \chpretitle{
%     заведующий кафедрой математических основ информатики и олимпиадного\\
%     программирования на базе МАОУ <<Ф"=Т лицей №1>>
% }
% \chtitle{г. Саратов, к.\,ф.-м.\,н., доцент}
% \chname{Кондратова\, Ю.\,Н.}

% Научный руководитель (для реферата преподаватель проверяющий работу)
\satitle{доцент} %должность, степень, звание
\saname{Г.\,Г.\,Наркайтис}

% Руководитель практики от организации (руководитель для цифровой кафедры)
\patitle{доцент, к.\,ф.-м.\,н.}
\paname{С.\,В.\,Миронов}

%Руководитель НИР
\nirtitle{} % степень, звание
\nirname{}

% Семестр (только для практики, для остальных типов работ не используется)
\term{2}

% Наименование практики (только для практики, для остальных типов работ не
% используется)
\practtype{учебная}

% Продолжительность практики (количество недель) (только для практики, для
% остальных типов работ не используется)
\duration{2}

% Даты начала и окончания практики (только для практики, для остальных типов
% работ не используется)
\practStart{01.07.2022}
\practFinish{13.01.2023}

% Год выполнения отчета
\date{2021}

\maketitle

% Включение нумерации рисунков, формул и таблиц по разделам (по умолчанию -
% нумерация сквозная) (допускается оба вида нумерации)
\secNumbering

\tableofcontents

% Раздел "Обозначения и сокращения". Может отсутствовать в работе
% \abbreviations
% \begin{description}
%     \item ... "--- ...
%     \item ... "--- ...
% \end{description}

% Раздел "Определения". Может отсутствовать в работе
% \definitions

% Раздел "Определения, обозначения и сокращения". Может отсутствовать в работе.
% Если присутствует, то заменяет собой разделы "Обозначения и сокращения" и
% "Определения"
% \defabbr

%\intro
\renewcommand\theFancyVerbLine{\small\arabic{FancyVerbLine}}
\setminted[cpp]{fontsize=\small,style=bw,linenos,breaklines}
\setminted[Makefile]{fontsize=\small,style=bw,linenos,breaklines}
\setminted[Python]{fontsize=\small,style=bw,linenos,breaklines}


\newpage
\subsection{Управление памятью на основе регионов}
\subsubsection{Мотивировка}
Текущая реализация абстрактного синтаксического дерева имеет следующие
недостатки:
\begin{enumerate}
    \item Выделение памяти стандартным методом может значительно
    фрагментировать оперативную память, затрудняя доступ к ней.
    \item Любое выделение и удаление памяти требует вмешательства системных
    вызовов, что может стать причиной дополнительных издержек во время
    работы программы.
    \item Программист не имеет возможности ручного управления выделяемой им
    памятью.
\end{enumerate}

Избавиться от этих недостатков можно используя различные оптимизации. В рамках
этой работы воспользуемся управлением памятью на основе, так называемых,
регионов (арен, зон) \cite{wangmemory}.

Под регионом далее будем понимать непрерывную область памяти, содержащую внутри
себя объекты. При запуске программы выделим регион некоторого размера, при
необходимости увеличивая его размер в некоторое постоянное число раз.

Этот подход имеет следующие преимущества:
\begin{enumerate}
    \item Элементы располагаются последовательно, в связи с чем минимизируется
    фрагментация и упрощается доступ к объектам.
    \item Выделение и освобождение памяти выполняется с минимальными
    издержками.
    \item Программисту предоставляется большая свобода для управления
    выделенной памятью.
\end{enumerate}
\subsubsection{Построение}
Формально определим требования к системе:
\begin{enumerate}
    \item Регион должен представлять из себя некоторый непрерывный участок
    размера $n$ байт (в начальный момент времени размер равен некоторой
    начальной величине $n_0$).
    \item При обращении к региону он должен предоставить $k$ байт памяти и
    вернуть некоторый идентификатор этого участка для последующего обращения.
    \item При заполнении региона должна быть возможность увеличить объем
    доступной памяти в некоторое число раз, которое далее будем называть
    коэффициентом увеличения.
    \item Должна быть доступна возможность эффективного освобождения всей
    выделенной регионом памяти.
\end{enumerate}

Единственной сложной операцией над регионом является его увеличение.
Так как выделение нового участка потенциально может сопровождаться изменением
адресов объектов, то необходимо организовать доступ к ним независимо
от первоначального адреса. Для этого для каждого объекта будем получать
доступ к нему через некоторый индекс.

Кроме того, коэффициент увеличения должен быть выбран таким образом, чтобы был
соблюден баланс между оптимальным объемом выделенной
памяти и частотой системных вызовов.
\subsubsection{Определение структуры}
Определим нашу структуру следующим образом:
\begin{minted}{cpp}
typedef struct arena {
    // Указатель на начало региона
    struct node* arena;
    // Размер региона
    unsigned int size;
    // Объем выделенной регионом памяти
    unsigned int allocated;
    } arena;
\end{minted}
\subsubsection{Инициализация}
Теперь определим функцию \verb|arena_construct|, выполняющую начальную
инициализацию состояния региона:

\begin{minted}{cpp}
int arena_construct (arena* arena) {
    // Начальный размер региона равен некоторой постоянной, равной DEFAULT_ARENA_SIZE
    arena->size = DEFAULT_ARENA_SIZE;
    arena->allocated = 0;
    // Выделим необходимое число памяти
    arena->arena = malloc(sizeof(node) * DEFAULT_ARENA_SIZE);
    // Если выделение прошло неудачно - вернем в качестве кода ошибки отличное от 0 значение.
    if (arena->arena == NULL) {
        return (!0);
    }
    return 0;
}
\end{minted}
\subsubsection{Выделение памяти}
После выделения некоторого объема памяти возможно обращение к ней. Определим
это обращение с помощью функции \verb|arena_allocate|:

\begin{minted}{cpp} 
int arena_allocate (arena* arena, unsigned int count) {
    // Если места в регионе недостаточно
    if (arena->allocated + count >= arena->size) {
        // Определим новый размер региона
        unsigned int newSize = MULTIPLY_FACTOR * arena->size;
        // Выделим регион большего размера и освободим ранее занятую память
        node* newArena = realloc(arena->arena, 
            newSize  * sizeof(node));
        if (NULL == newArena) {
            return -1;
        }
        arena->arena = newArena;
        arena->size = newSize;
    }
    // В качестве результата вернем индекс первого свободного участка региона
    unsigned int result = arena->allocated;
    // Сместим индекс на объем выделенной памяти
    arena->allocated += count;
    // Вернем результат
    return result;
}
\end{minted}

Отметим, что наиболее часто значением \verb|MULTIPLY_FACTOR| оказываются числа
1.5 и 2. Это позволяет достичь амортизационно константного времени выполнения
операции выделения памяти \cite{facebook}.
\subsubsection{Освобождение выделенной памяти}
Наконец, реализуем освобождение выделенной региону памяти с помощью функции
\verb|arena_free|:

\begin{minted}{cpp}
void arena_free (arena* arena) {
    if (arena->arena != NULL)
        free(arena->arena);
    arena->arena = NULL;
}
\end{minted}
\subsubsection{Модификация абстрактного синтаксического дерева}
Осталось изменить исходный код программы, чтобы обеспечить выделение памяти с
помощью полученной нами структуры данных.

Для этого воспользуемся директивой \verb|%param| и заявим в качестве параметра
переменную типа \verb|arena*|. В функциях \verb|eval|, \verb|newnum|,
\verb|newast| внесем изменения, чтобы обеспечить выделение памятью с помощью
написанных ранее функций.

С полным кодом программы можно ознакомиться в приложении \ref{app:A}.
\subsubsection{Сборка проекта}
Теперь проект можно собрать, незначительно изменив \verb|Makefile|:
\inputminted{Makefile}{Makefile} 
и запустить. Результат работы программы представлен на рис. 6
\newpage
\section{Сравнение полученных реализаций}
Проведем анализ производительности полученных версий анализатора. В
качестве данных для тестирования возьмем выражения вида 
$\underbrace{2+2+2 \dots +2}_n$ для $n = 1 \dots 100$ с шагом 1. 
Для вычисления времени выполнения воспользуемся библиотекой \verb|time|
Python 3.9.5. Автоматизацию обеспечим с помощью библиотеки \verb|subprocess|.
Получим следующий код:
\inputminted{Python}{test.py}
Кроме того, отметим, что в ранее написанные программы были внесены некоторые
изменения для проведения эксперимента. Ознакомиться с ними
можно в приложении \ref{app:A}.

Ознакомиться с полным исходным кодом программы, осуществляющей
исследование производительности можно в приложении \ref{app:B}.

Для большей наглядности графики интерполированы полиномом с помощью функции
\verb|polyfit| библиотеки \verb|numpy|.

Ознакомиться с полным исходным кодом программы, осуществляющей
анализ полученных результатов можно в приложении \ref{app:C}.
Результаты исследования изображены на рис. 7:

Исследование показало, что использование абстрактных синтаксических
деревьев позволяет уменьшить время работы программы более чем в 5 раз, что
существенно заметно для выражений любой длины.

Также из графиков видно, что в рамках данной работы не удалось добиться большей
производительности при управлении памятью на основе регионов.
Тем не менее, она все еще может считаться более предпочительной ввиду
перечисленных ранее преимуществ.


\newpage
\conclusion
В ходе данной работы:
\begin{enumerate}
    \item Были изучены теоретические основы построения лексических и
    синтаксических анализаторов.
    \item Проанализированы особенности реализации лексических и синтаксических
    анализаторов.
    \item Были изучены принципы работы генераторов лексического и
    синтаксического анализа на примере Flex и GNU Bison.
    \item Были созданы лексический и синтаксический анализаторы для анализа
    математического выражения.
    \item Было изучено понятие абстрактного синтаксического дерева.
    \item Проведен анализ производительности полученных реализаций.
\end{enumerate}

Таким образом, все поставленные в рамках работы задачи выполнены.

Результаты исследования показали, что абстрактные синтаксические деревья
позволяют добиться увеличения производительности в 5--6 раз.

А это, в свою очередь, позволяет утверждать о том, что концепция абстрактных
синтаксических деревьев является крайне важной в информатике и
ее приложениях, в частности, при создании синтаксических анализаторов.
\newpage

% Отобразить все источники. Даже те, на которые нет ссылок.
%\nocite{*}

% Меняем inputencoding на лету, чтобы работать с библиографией в кодировке
% `cp1251', в то время как остальной документ находится в кодировке `utf8'
% Credit: Никита Рыданов
\inputencoding{cp1251}
\bibliographystyle{gost780uv}
\bibliography{thesis}
\inputencoding{utf8}

% Окончание основного документа и начало приложений Каждая последующая секция
% документа будет являться приложением
\appendix
\section{Flash"=носитель с исходным кодом программ, использующихся в работе}
\label{app:A}
Папка \verb|src| содержит оригинальный исходный код программы:

Папка \verb|naive| "--- реализация без АСД

Папка \verb|naiveast| "--- реализация с АСД

Папка \verb|arena| "--- реализация с АСД на основе региона

Папка \verb|extsrc| содержит измененный исходный код, необходимый для исследования производительности:

Папка \verb|naive| "--- реализация без АСД

Папка \verb|naiveast| "--- реализация с АСД

Папка \verb|arena| "--- реализация с АСД на основе региона
\section{Исходный код программы на Python, осуществляющей исследование
производительности полученных реализаций}
\label{app:B}
\inputminted{Python}{test.py}
\section{Исходный код программы на Python, осуществляющей анализ
полученных результатов}
\label{app:C}
\inputminted{Python}{graph.py}

\end{document}