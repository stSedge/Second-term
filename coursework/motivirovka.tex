Текущая реализация абстрактного синтаксического дерева имеет следующие
недостатки:
\begin{enumerate}
    \item Выделение памяти стандартным методом может значительно
    фрагментировать оперативную память, затрудняя доступ к ней.
    \item Любое выделение и удаление памяти требует вмешательства системных
    вызовов, что может стать причиной дополнительных издержек во время
    работы программы.
    \item Программист не имеет возможности ручного управления выделяемой им
    памятью.
\end{enumerate}

Избавиться от этих недостатков можно используя различные оптимизации. В рамках
этой работы воспользуемся управлением памятью на основе, так называемых,
регионов (арен, зон) \cite{wangmemory}.

Под регионом далее будем понимать непрерывную область памяти, содержащую внутри
себя объекты. При запуске программы выделим регион некоторого размера, при
необходимости увеличивая его размер в некоторое постоянное число раз.

Этот подход имеет следующие преимущества:
\begin{enumerate}
    \item Элементы располагаются последовательно, в связи с чем минимизируется
    фрагментация и упрощается доступ к объектам.
    \item Выделение и освобождение памяти выполняется с минимальными
    издержками.
    \item Программисту предоставляется большая свобода для управления
    выделенной памятью.
\end{enumerate}