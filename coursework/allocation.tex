После выделения некоторого объема памяти возможно обращение к ней. Определим
это обращение с помощью функции \verb|arena_allocate|:

\begin{minted}{cpp} 
int arena_allocate (arena* arena, unsigned int count) {
    // Если места в регионе недостаточно
    if (arena->allocated + count >= arena->size) {
        // Определим новый размер региона
        unsigned int newSize = MULTIPLY_FACTOR * arena->size;
        // Выделим регион большего размера и освободим ранее занятую память
        node* newArena = realloc(arena->arena, 
            newSize  * sizeof(node));
        if (NULL == newArena) {
            return -1;
        }
        arena->arena = newArena;
        arena->size = newSize;
    }
    // В качестве результата вернем индекс первого свободного участка региона
    unsigned int result = arena->allocated;
    // Сместим индекс на объем выделенной памяти
    arena->allocated += count;
    // Вернем результат
    return result;
}
\end{minted}

Отметим, что наиболее часто значением \verb|MULTIPLY_FACTOR| оказываются числа
1.5 и 2. Это позволяет достичь амортизационно константного времени выполнения
операции выделения памяти \cite{facebook}.