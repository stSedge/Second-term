\documentclass{article}
\usepackage[T2A]{fontenc}
\usepackage[utf8]{inputenc}
\usepackage{amsthm}
\usepackage{amsmath}
\usepackage{amssymb}
\usepackage{amsfonts}
\usepackage{mathrsfs}
\usepackage[12pt]{extsizes}
\usepackage{fancyvrb}
\usepackage{indentfirst}
\usepackage[
  left=2cm, right=2cm, top=2cm, bottom=2cm, headsep=0.2cm, footskip=0.6cm,
  bindingoffset=0cm]{geometry}
\usepackage[english,russian]{babel}


\begin{document}
\setlength{\abovedisplayskip}{7pt}
\setlength{\belowdisplayskip}{7pt}
\section*{Вариант 10}
Решение уравнений с учетом краевых условий представим в виде
\begin{equation}
    \label{eq:W}
    W =  \sum\limits_{k=1}^{\infty} \left[\left(R_k(\tau)+R_k^0
    \right)\cos\frac{(2k - 1)\pi\xi}{2}+Q_k(\tau)\sin
    k\pi\xi\right].
\end{equation}
Верхний индекс 0 в \eqref{eq:W} означает решение, соответствующее постоянному
уровню давления $p_0$, независящему от $\tau$.

Принимая во внимание линейность предыдущего уравнения и подставляя в него
\eqref{eq:W}, найденное выражение для давления, а, также раскладывая оставшиеся
члены, входящие в его правую часть в ряды по тригонометрическим функциям, из
полученного уравнения запишем выражение для составляющей, независящей от
времени $R_k^0$
\begin{equation*}
    R_k^0 = (2\ell/((2k - 1)\pi))^4\left(4(-1)^{k + 1}/((2k - 1)\pi)\right)
    p_0(Dw_m)^{-1},
\end{equation*}
и уравнения для определения $R_k(\tau)$ и $Q_k(\tau)$
\begin{equation}
    a_{1ck}w_mR_k+a_{2ck}w_mdR_k/d\tau = 2(-1)^{k + 1}/((2k - 1)\pi D)
    p_m^+f_p(\tau),
\end{equation}
\begin{equation}
    a_{1sk}w_mQ_k+a_{2sk}w_mdQ_k/d\tau = (-1)^{k + 1}/(k\pi D)p_m^+f_p(\tau).
\end{equation}

\end{document}