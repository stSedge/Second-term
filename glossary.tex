\documentclass[14pt]{extarticle}

\usepackage[T2A]{fontenc}
\usepackage[english,russian]{babel}
\usepackage[utf8]{inputenc}
\usepackage{geometry}
\geometry{a4paper,top=2cm,bottom=2cm,left=2cm,right=2cm}

\begin{document}
\begin{flushright}
    Стюхина Ангелина, 151 гр. КНиИТ
\end{flushright}
\begin{center}
    \Large \textbf{Глоссарий}
\end{center}
\begin{enumerate}
    \section*{A}
    \item \textbf{Agile (Agile software development)} (\textit{Никита Барабанов}) "--- метод организации рабочего процесса, при котором разработка всего проекта делится на большое количество мелких шагов. Выполнение каждого из них называется спринтом.
    \section*{B}
    \item \textbf{Backend"=тестирование} (англ. \textit{backend testing}) (\textit{Павел Пасеков}) "---
    метод проверки корректности работы серверных компонентов приложения,
    включая обработку запросов, аутентификацию и авторизацию пользователя, обработку ошибок и так далее.

    \item \textbf{Big data} (\textit{Алексей Кузьмин}) "--- чрезвычайно крупные и сложные наборы данных, которые не
    поддаются простому управлению или анализу с использованием традиционных методов обработки данных.
    Эти наборы данных могут включать структурированные или неструктурированные данные из различных
    источников, таких как социальные сети, датчики, мобильные устройства и другие цифровые платформы.

    \item \textbf{Business intelligence (BI)} (\textit{Алексей Кузьмин}) "--- поиск оптимальных бизнес"=решений с помощью обработки большого объема информации неструктурированных данных для полноты картины бизнеса и принятия правильных операционных и стратегических решений.
    \section*{C}

    \item \textbf{ChatGPT (Chat Generative Pre-trained Transformer)} (\textit{Никита Рыданов}) "--- 
    языковая модель на базе искусственного интеллекта, разработанная компанией \textit{OpenAI}. Она была обучена на
    огромном количестве текстовых данных из Интернета и может генерировать текстовые ответы, подобные
    человеческим, на заданную подсказку.
    \section*{D}
    \item \textbf{Data mining} (\textit{Алексей Кузьмин}) "--- междисциплинарная область, возникшая и развивающаяся
    на базе таких наук, как прикладная статистика, распознавание образов, искусственный интеллект, теория
    баз данных и других.

    \item \textbf{Data sience (DS)} (\textit{Алексей Кузьмин}) "--- междисциплинарная область на стыке статистики, математики, системного анализа
    и машинного обучения, которая охватывает все этапы работы с данными. Она предполагает исследование и анализ сверхбольши
    массивов информации и ориентирована на получение практических результатов.

    \item \textbf{Deep learning} (\textit{Алексей Кузьмин}) "--- вид машинного обучения с использованием многослойных нейронных сетей, которые
    самообучаются на большом наборе данных.
    \section*{E}
    \item \textbf{Embedding} (\textit{Михаил Чернигин}) "---  результат процесса преобразования языковой сущности: слова, предложения, параграфа или целого текста "--- в набор чисел "--- числовой вектор.
    
    \item \textbf{ERP-системы} (\textit{Ростислав, Сибинтек}) (\textit{англ. enterprise resource planning}) "--- совокупность всех базовых бизнес-процессов, необходимых для управления компанией: финансы, управление персоналом, производство, цепочка поставок, услуги, закупки и многое другое.
    \section*{M}
    
    \item \textbf{Machine Learning (ML)} (\textit{Алексей Кузьмин}) "--- направление искусственного интеллекта, сосредоточенное на создании систем, которые обучаются и развиваются на основе получаемых ими данных.
    \section*{O}
    \item \textbf{Open"=source} (\textit{Никита Рыданов}) "--- программное обеспечение, распространяемое с открытым исходным кодом. Такое приложение можно доработать под свои задачи без нарушения авторских прав разработчиков, а также изучить на наличие уязвимостей, использовать для разработки других программ и так далее.
    \section*{P}
    \item \textbf{Pipeline} (\textit{Никита Барабанов}) "--- последовательные стадии преобразования данных, предшествующие их загрузке в модель.
    
    \item \textbf{Proxy} (\textit{Игорь Юрин}) "--- сервер-посредник между пользователем и интернет-ресурсом.
    \section*{R}
    
    \item \textbf{Roadmap} (\textit{Павел Пасеков}) "--- документ, в котором перечислены цели проекта, его ключевые этапы, контрольные даты и ответственные исполнители.
    \section*{S}
    \item \textbf{SOA (Service-oriented architecture)} (\textit{Никита Барабанов}) "--- метод разработки программного обеспечения, который использует программные компоненты, называемые сервисами, для создания бизнес"=приложений.
    
    \item \textbf{SRE (Site reliability engineering)} (\textit{Павел Пасеков}) "--- сфера обеспечения бесперебойной работы высоконагруженных сервисов.
    \section*{T}
    
    \item \textbf{Teamlead} (\textit{Павел Пасеков}) "--- специалист, который руководит командой разработчиков.
    \section*{U}
    \item \textbf{UI/UX"=дизайн (user interface and user experience)} (\textit{Алексей Кузьмин}) "--- проектирование удобных, понятных и эстетичных пользовательских интерфейсов.
    \item \textbf{Unreal Engine (UE)} (\textit{Леонид Сорокин}) "--- движок для создания игр, один из двух наиболее популярных в мире.
    \section*{А}
    \item \textbf{Адаптер} (\textit{Никита Барабанов}) "--- аппаратное устройство или программный компонент, преобразующий передаваемые данные из одного представления в другое.
    \item \textbf{Алгоритм} (\textit{Алексей Кузьмин}) "--- способ с точным (т.е. выраженным в точно
    определенном языке) конечным описанием применения практически выполнимых элементарных шагов переработки информации.
    \item \textbf{Антивирус} (\textit{Игорь Юрин}) "--- специализированная программа для обнаружения компьютерных вирусов и их удаления. Также может выполняться профилактика в виде обнаружения и прекращения выполнения нежелательных программ и регулярная проверка файлов компьютера для выявления подозрительных файлов.
    \item \textbf{Аппроксимация}  (\textit{Алексей Кузьмин}) "--- моделирование сложной функции более простой с вычислительной точки зрения функцией,
    имеющей минимальные отклонения от исходной в заданной области.
    \item \textbf{Асессор} (\textit{Никита Рыданов}) "--- специалист, который проверяет релевантность страниц в выдаче тем поисковым запросам, которые вводят пользователи, для улучшения работы поисковой машины.
    \item \textbf{Аутсорсинг} (\textit{англ. outsourcing}) (\textit{Ростислав, Сибинтек}) "--- процесс передачи непрофильных задач или функций компании на выполнение внешней стороне, обычно другой компании или поставщику услуг.
    \section*{Б}
    \item \textbf{Баг} (\textit{англ. bug}) (\textit{Никита Барабанов}) "--- ошибка в программе, реже в аппаратной части (железе).
    \item \textbf{База данных (БД)} (\textit{англ. database}) (\textit{Никита Барабанов}) "--- упорядоченный набор структурированной информации или данных, которые обычно хранятся в электронном виде в компьютерной системе.
    \item \textbf{Бэклог продукта} (англ. \textit{product backlog}) (\textit{Никита Барабанов}) "--- 
    упорядоченный и постоянно обновляемый список всего, что планируется сделать для создания и улучшения продукта.
    \section*{В}
    \item \textbf{Вирус} (\textit{Игорь Юрин}) "--- вид вредоносного программного обеспечения, которое способно
    распространять свои копии с целью заражения и повреждения данных на устройстве жертвы.
    \section*{Г}
    \item \textbf{Генеративно-состязательная нейросеть} \textit{(англ. generative adversarial network, GAN)} 
    (\textit{Алексей Кузьмин}) "--- структура машинного обучения, состоящая из двух нейронных сетей,
    одна из которых обучена генерировать данные, а другая "--- отличать смоделированные данные от реальных, конкурирующих за получение более точных прогнозов, таких как изображения, уникальная музыка и так далее.

    \item \textbf{Граф} (\textit{Галина Громова}) "--- система объектов произвольной природы (\textit{вершин}) и связок (\textit{ребер}), соединяющих некоторые пары этих объектов.
    
    \section*{Д}
    \item \textbf{Деплой} (\textit{англ. deploy}) (\textit{Никита Барабанов}) "--- размещение готовой версии программного обеспечения на платформе, доступной для пользователей.
    \item \textbf{Докер} (\textit{англ. docker}) (\textit{Алексей Кузьмин}) "--- одновременно платформа и технология для контейнеризации. Она позволяет создавать контейнеры и управлять ими для развертывания и доставки кода на сервер.
    \section*{З}
    \item \textbf{Запушить} (\textit{англ. to push}) (\textit{Никита Барабанов}) "--- отправить код в репозиторий (с помощью команды \textit{push}).
    \item \textbf{Защита информации} (\textit{Игорь Юрин}) "--- совокупность мероприятий, направленных на обеспечение конфиденциальности и целостности обрабатываемой информации, а также доступности информации для пользователей.
    \section*{И}
    \item \textbf{Игровой движок} (\textit{англ. game engine}) (\textit{Леонид Сорокин}) "--- программный комплекс, который упрощает разработку игр, предоставляя набор необходимых для разработки инструментов.
    \item \textbf{Интерфейс} (\textit{англ. interface}) (\textit{Никита Барабанов}) "--- набор инструментов, который позволяет пользователю взаимодействовать с программой. В более широком смысле термин обозначает любые инструменты для соприкосновения между разными системами и сущностями.
    \item \textbf{Информационная безопасность (ИБ)} (\textit{Игорь Юрин}) "--- защита любых данных компании вне зависимости от их формы. Цель этой практики "--- обеспечить конфиденциальность, целостность и доступность информации.
    \section*{К}
    \item \textbf{Классификация} (\textit{Алексей Кузьмин}) "--- процесс группирования объектов по категориям на основе предварительно классифицированного тренировочного набора данных. Относится к алгоритмам контролируемого машинного обучения.
    \item \textbf{Кластеризация} (\textit{Галина Громова}) "--- задача неконтролируемого машинного обучения, которая группирует отдельные экземпляры данных в кластеры со сходными характеристиками.
    \item \textbf{Компонента сильной (слабой) связности} (\textit{Галина Громова}) "--- максимальное множество вершин ориентированного графа, между любыми двумя вершинами которого существует путь по дугам (без учета направления).
    \item \textbf{Компьютерная безопасность (КБ)} (\textit{Игорь Юрин}) "--- одно из направлений информационной безопастности. Это меры по защите от киберугроз компьютеров, серверов, сетей, электронных систем, устройств и приложений.
    \item \textbf{Конечный автомат} (\textit{Леонид Сорокин}) "--- математическая абстракция, модель дискретного устройства, имеющего один вход, один выход и в каждый момент времени находящегося в одном состоянии из множества возможных.
    \section*{Л}
    \item \textbf{Лог"=файл} (\textit{англ. log}) (\textit{Никита Барабанов}) "--- специальный журнал (файл), в котором хранится информация о состоянии работы приложения или программы.
    \section*{М}
    \item \textbf{Микросервисная архитектура} (\textit{Алексей Кузьмин}) "--- методология разработки, которая предполагает создание приложения из набора слабо связанных сервисов, каждый из которых может быть разработан, тестирован и развернут независимо от других.
    \item \textbf{Многослойный персептрон} (\textit{Алексей Кузьмин}) "--- нейронная сеть прямого распространения сигнала (без обратных связей), в которой входной сигнал преобразуется в выходной, проходя последовательно через несколько слоев.
    \item \textbf{Модульное тестирование} (\textit{Максим, КБПА}) "--- тип тестирования программного обеспечения, при котором тестируются отдельные модули или компоненты программного обеспечения. Его цель заключается в том, чтобы проверить, что каждая единица программного кода работает должным образом.
    \item \textbf{Монолитная архитектура} (\textit{Никита Барабанов}) "--- традиционная модель программного обеспечения, которая представляет собой единый модуль, работающий автономно и независимо от других приложений.
    \section*{Н}
    \item \textbf{Нейронная сеть} (\textit{Алексей Кузьмин}) "--- компьютерная система, имитирующая работу нервной системы человека. Она используется для решения различных задач машинного обучения, таких как распознавание образов и прогнозирование. Нейронные сети могут обучаться на основе большого количества данных и оптимизировать свои параметры для достижения наилучшей производительности.
    \section*{П}
    \item \textbf{Паттерны программирования} (\textit{Иван Жадаев}) "--- способы построения программ, которые считаются <<хорошим тоном>> для разработчиков. Их еще называют шаблонами или образцами: чаще всего паттерн "--- это типовое решение для часто встречающейся задачи на построение.
    \item \textbf{Пиксель} (\textit{англ. pixel}) (\textit{Алена Коноплева}) "--- одна из множества точек, составляющих изображение на экране электронного устройства, а также наименьшая единица растровой графики.
    \item \textbf{Программное обеспечение (ПО)} (\textit{Ростислав, Сибинтек}) "--- совокупность всех программ на персональном компьютере. Реже так называют и сами программы. 
    \item \textbf{Протокол} (\textit{Никита Барабанов}) "--- набор соглашений интерфейса логического уровня, которые определяют обмен данными между различными программами.
    \section*{Р}
    \item \textbf{Рекуррентные сети} (\textit{англ. recurrent neural network, RNN}) (\textit{Алексей Кузьмин}) "--- сети с обратными или перекрестными связями между различными слоями нейронов.
    \item \textbf{Релизный цикл} (\textit{Никита Барабанов}) "--- период времени, за который новый функционал проходит путь от принятия решения о его реализации до выхода в продуктивную версию приложения.
    \item \textbf{Репозиторий} (\textit{англ. repository}) (\textit{Никита Барабанов}) "--- локальная папка, где хранятся версии проекта, или глобальное хранилище на удаленном сервисе, например, на \textit{GitHub}.
    \item \textbf{Рефакторинг} (\textit{англ. refactoring}) (\textit{Никита Барабанов}) "--- изменение кода, которое не затрагивает его функциональность, но улучшает читаемость и упрощает дальнейшую поддержку.
    \section*{С}
    \item \textbf{Сверточные нейронные сети (СНС)} (\textit{англ. Convolutional Neural Networks, CNN}) (\textit{Алексей Кузьмин}) "--- класс нейронных сетей, который специализируется на обработке изображений и видео. Такие нейросети хорошо улавливают локальный контекст, когда информация в пространстве непрерывна, то есть ее носители находятся рядом.
    \item \textbf{Сервер} (\textit{Никита Барабанов}) "--- сетевой компьютер, обрабатывающий запросы от других компьютеров в локальной или глобальной сети.
    \item \textbf{Скрам"=доска} (\textit{англ. scrum board}) (\textit{Никита Барабанов}) "--- визуальная презентация технических задач, которые должны быть решены за один спринт.
    \section*{Т}
    \item \textbf{Трансформер} (\textit{англ. transformer}) (\textit{Никита Рыданов}) "--- тип нейронной сети, который направлен на решение последовательностей с обработкой зависимостей. Таким образом, трансформеры не обрабатывают последовательности по порядку "--- они способны сразу фокусироваться на необходимых элементах данных благодаря <<механизму внимания>>.
    \item \textbf{Троян} (\textit{англ. trojan}) (\textit{Игорь Юрин})"--- разновидность вредоносной программы, проникающей в компьютер под видом легитимного программного обеспечения, в отличие от вирусов и червей,
    которые распространяются самопроизвольно.
    \section*{Ф}
    \item \textbf{Фиксить} (\textit{англ. to fix}) (\textit{Никита Барабанов})"--- исправлять баг.
    \item \textbf{Фреймворк} (\textit{англ. framework}) (\textit{Никита Барабанов}) "--- программная оболочка с набором готовых решений (методов, блоков, функциональности), позволяющая упростить и ускорить решение типовых задач, характерных для языка программирования.
    \section*{Ч}
    \item \textbf{Черный ящик} (\textit{Максим, КБПА}) "--- система, внутреннее устройство и алгоритм функционирования которой весьма трудны, непонятны или маловажны внутри некоторой задачи.
    \section*{Ю}
    \item \textbf{Юзер} (\textit{англ. user}) (\textit{Никита Барабанов}) "--- пользователь.
\end{enumerate}
\end{document}